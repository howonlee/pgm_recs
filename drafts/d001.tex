Dear JH:
(PGM tutorial is in pgm_tutorial repo)

What's original in this is not mine and what's mine is not original. I hope I've sourced everything enough.

Consider a social network. It has strange properties. Fat tails in indegrees and outdegrees, clustering coefficients, synchronizability, percolation possibility and huge component.

The reason why it has such properties is because of the underlying generative process that creates them. There are many possibilities, but the essential thing about them is positive feedback, from BA to SKG to RTG to Chung Lu.

Actually, this is very close to a well-disseminated idea by NN Taleb and BB Mandelbrot, which is that of the Black Swan. This is because of the distributional structure of the world which he advocates. The human social network is actually of a black swan nature. SKG is actually a fractal.

One way to attack human social networks is by percolation graph matching. This was invented by Narayanan as far as I can tell to de-anonymize social networks. It is a special case of graph matching.

The reason why it's interesting is that it is fast in comparison to less easy graph matching, although you can do other graph matching methods. Moreover, we just showed that the graph is easy to percolate on.

Tutorial on percolation graph matching.

The Black Swan can be attacked by percolation graph matching and the creation of an ansatz.

The reason why this is an attractive method is that one node can have an effect on a long distance from another node viably. Depending on r previously matched nodes to spread, but those previously matched nodes in turn may have depended on some other nodes to spread, and on, until to get to another node which may be far away. More formally, a critical phase transition has been proved. This appeals to me intuitively a fair bunch.

Some speculations on what you can do with other possible signals. Ubiquity of power law sort of things in other domains, although you must be very careful. Spectral properties of financial and turbulent liquid signals, software networks, biological networks, economic networks are also of interest.

Probably a most important speculation is on the human brain. A fairly heterodox view about lognormal distributions being general in human cognition exists, but all I really think about is that an observed power law inheres in human reward discounting and in perception. I note that it will be possible to fit together PGM with, say, a Boltzmann machine or MLP core, and that may be interesting. It's my next project, anyhow, and it seems highly related to the 2009 SA attack - I mentioned Boltzmann machine first because of this. I do not think that graph matching is such a fundamentally different problem from constraint satisfaction that a matching cannot be found.

Here's a thing that may not have anything to do with anything. It's from J. Crutchfield of UC Davis and the Santa Fe Institute and it's been stuck in my head for quite the while.
